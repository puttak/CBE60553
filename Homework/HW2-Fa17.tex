% Created 2017-09-01 Fri 21:26
% Intended LaTeX compiler: pdflatex
\documentclass[11pt]{article}
\usepackage[utf8]{inputenc}
\usepackage{lmodern}
\usepackage[T1]{fontenc}
\usepackage{fixltx2e}
\usepackage{graphicx}
\usepackage{longtable}
\usepackage{float}
\usepackage{wrapfig}
\usepackage{rotating}
\usepackage[normalem]{ulem}
\usepackage{amsmath}
\usepackage{textcomp}
\usepackage{marvosym}
\usepackage{wasysym}
\usepackage{amssymb}
\usepackage{amsmath}
\usepackage[version=3]{mhchem}
\usepackage[numbers,super,sort&compress]{natbib}
\usepackage{natmove}
\usepackage{url}
\usepackage{minted}
\usepackage{underscore}
\usepackage[linktocpage,pdfstartview=FitH,colorlinks,
linkcolor=blue,anchorcolor=blue,
citecolor=blue,filecolor=blue,menucolor=blue,urlcolor=blue]{hyperref}
\usepackage{attachfile}
\usepackage[left=1in, right=1in, top=1in, bottom=1in, nohead]{geometry}
\geometry{margin=1.0in}
\usepackage{hyperref}
\usepackage{amsmath}
\usepackage{graphicx}
\usepackage{epstopdf}
\usepackage{fancyhdr}
\pagestyle{fancy}
\fancyhf{}
\usepackage[labelfont=bf]{caption}
\usepackage{setspace}
\setlength{\headheight}{10.2pt}
\setlength{\headsep}{20pt}
\renewcommand{\headrulewidth}{0.5pt}
\renewcommand{\footrulewidth}{0.5pt}
\lfoot{\today}
\cfoot{\copyright\ 2016 W.\ F.\ Schneider}
\rfoot{\thepage}
\chead{\bf{Advanced Chemical Engineering Thermodynamics (CBE 60553)\vspace{12pt}}}
\lhead{\bf{Homework 2}}
\rhead{\bf{Due September 11, 2017}}
\usepackage{titlesec}
\titlespacing*{\section}
{0pt}{0.6\baselineskip}{0.2\baselineskip}
\title{University of Notre Dame\\Advanced Chemical Engineering Thermodynamics\\(CBE 60553)}
\author{Prof. William F.\ Schneider}
\usepackage{siunitx}
\usepackage[version=3]{mhchem}
\def\dbar{{\mathchar'26\mkern-12mu d}}
\setcounter{secnumdepth}{3}
\author{William F. Schneider}
\date{\today}
\title{CBE 60553 Homework}
\begin{document}

\begin{OPTIONS}
\end{OPTIONS}

\noindent \textbf{Solve each problem on separate sheets of paper, and clearly indicate the problem number and your name on each.  Carefully and neatly document your answers.  You may use a mathematical solver like Matlab or Mathematica. Use plotting software for all plots.}

\section{The fundamental equation knows all, sees all}
\label{sec:org8910be7}
A system is described by the fundamental equation:
\begin{equation}
U=\left (\frac{v_0 \theta}{R^2} \right ) \frac{S^3}{NV}\label{eq:1}
\end{equation}
where \(R\), \(\theta\), and \(v_0\) are all positive constants.

\begin{enumerate}
\item Find the three equations of state of the system

\item Convince yourself (and me) that the three equations are first-order homogeneous and that \(T(S,V,N)\) is intrinsically positive.

\item Plot the ``adiabats'' (loci of constant entropy) in the \(P-v\) plane.

\item Find the ``mechanical equation of state'' \(P=P(T,v)\) of the system and
plot the corresponding isotherms (loci of constant temperature) in the \(P-V\) plane.
\end{enumerate}

\section{But it reveals itself through its equations of state}
\label{sec:org103d9cb}
A particular system is described by the thermal and mechanical equations of state below, where \(A\) is a
  positive constant.
\begin{equation*}
  T = \frac{3 A s^2}{v}\ \ \ \ \ \ \ \ P = \frac{A s^3}{v^2}
\end{equation*}

\begin{enumerate}
\item Find the corresponding chemical potential \(\mu(s,v)\).
\item Find the corresponding fundamental equation \(u(s,v)\).
\end{enumerate}

\section{Relax that constraint}
\label{sec:org32637a1}
Equations of state of two systems are given below.  System 1 contains 2 moles and is
  at \SI{250}{K}.  System 2 contains 3 moles and is \SI{350}{K}.  Suppose the two systems are
  connected by a diathermal wall.  What are the energies and temperatures of the two
  systems after equilibrium is established?

\begin{equation*}
  U^{(1)}=\frac{3}{2} R n^{(1)} T^{(1)}\ \ \ \ \ \ \ \  U^{(2)}=\frac{5}{2} R n^{(2)} T^{(2)}\
\end{equation*}

\section{Work with an ideal gas}
\label{sec:orgfb187c7}
When an ideal monatomic gas is placed inside a piston and compressed
  quasi-statically, the temperature and volume are observed to vary according
  to
\begin{equation*}
  T=\left ( \frac{V}{V_0}\right )^\eta T_0
\end{equation*}
where \(\eta\) is a constant.

\begin{enumerate}
\item Calculate the work \(w\) done on the gas when one mole is compressed from \(V_0\) to
\(V_1 < V_0\).

\item Calculate the corresponding change in energy \(\Delta U\) of the gas.

\item Calculate the corresponding heat transfered to the gas \(q\) by energy balance.

\item Calculate the heat transfered to the gas \(q\) by integrating \(\dbar q = T dS\).

\item Calculate the value of \(\eta\) for which \(q = 0\).
\end{enumerate}
\section{Maximum work, minimum work}
\label{sec:org7de9d5f}
One mole of a monatomic ideal gas is contained in a cylinder of volume
  \(10^{-3}~\text{m}^3\) at \SI{400}{K}.  The gas is to be brought to a final state of
  twice the volume and the same temperature.

\begin{enumerate}
\item What is the maximum work that can be delivered to a reversible work
source, assuming a \SI{300}{K} thermal reservoir is available?

\item One way to extract this work would be to expand the gas adiabatically to
\SI{300}{K}, compress it isothermally at \SI{300}{K}, and finally compress it
adiabatically to the final state.  Find the work and heat transfers in each step.

\item What is the minimum work necessary to return the gas to its initial
state, assuming the same \SI{300}{K} reservoir is available?
\end{enumerate}
\end{document}