% Created 2017-09-30 Sat 10:37
% Intended LaTeX compiler: pdflatex
\documentclass[11pt]{article}
\usepackage[utf8]{inputenc}
\usepackage{lmodern}
\usepackage[T1]{fontenc}
\usepackage{fixltx2e}
\usepackage{graphicx}
\usepackage{longtable}
\usepackage{float}
\usepackage{wrapfig}
\usepackage{rotating}
\usepackage[normalem]{ulem}
\usepackage{amsmath}
\usepackage{textcomp}
\usepackage{marvosym}
\usepackage{wasysym}
\usepackage{amssymb}
\usepackage{amsmath}
\usepackage[version=3]{mhchem}
\usepackage[numbers,super,sort&compress]{natbib}
\usepackage{natmove}
\usepackage{url}
\usepackage{minted}
\usepackage{underscore}
\usepackage[linktocpage,pdfstartview=FitH,colorlinks,
linkcolor=blue,anchorcolor=blue,
citecolor=blue,filecolor=blue,menucolor=blue,urlcolor=blue]{hyperref}
\usepackage{attachfile}
\usepackage[left=1in, right=1in, top=1in, bottom=1in, nohead]{geometry}
\geometry{margin=1.0in}
\usepackage{hyperref}
\usepackage{amsmath}
\usepackage{graphicx}
\usepackage{epstopdf}
\usepackage{fancyhdr}
\pagestyle{fancy}
\fancyhf{}
\usepackage[labelfont=bf]{caption}
\usepackage{setspace}
\setlength{\headheight}{10.2pt}
\setlength{\headsep}{20pt}
\renewcommand{\headrulewidth}{0.5pt}
\renewcommand{\footrulewidth}{0.5pt}
\lfoot{\today}
\cfoot{\copyright\ 2017 W.\ F.\ Schneider}
\rfoot{\thepage}
\chead{\bf{Advanced Chemical Engineering Thermodynamics (CBE 60553)\vspace{12pt}}}
\lhead{\bf{Homework 4}}
\rhead{\bf{Due October 2, 2017}}
\usepackage{titlesec}
\titlespacing*{\section}
{0pt}{0.6\baselineskip}{0.2\baselineskip}
\title{University of Notre Dame\\Advanced Chemical Engineering Thermodynamics\\(CBE 60553)}
\author{Prof. William F.\ Schneider}
\usepackage{siunitx}
\usepackage[version=3]{mhchem}
\def\dbar{{\mathchar'26\mkern-12mu d}}
\setcounter{secnumdepth}{3}
\author{William F. Schneider}
\date{\today}
\title{CBE 60553 Homework}
\begin{document}

\begin{OPTIONS}
\end{OPTIONS}

\noindent \textbf{Solve each problem on separate sheets of paper, and clearly indicate the problem number and your name on each.  Carefully and neatly document your answers.  You may use a mathematical solver like Matlab or Mathematica. Use plotting software for all plots.}

\section{van der Waals \ce{CO2} redux}
\label{sec:org343edfc}
Last homework you worked with the fundamental van der Waals equation:
\begin{equation}
s_\text{vdW}(u,v)=s_{0}+R\ln\left (v-b\right ) +c R \ln \left ( u+a/v \right )
\end{equation}
\begin{enumerate}
\item Construct the Helmholtz potential of a van der Waals fluid,
\(a_\text{vdW}(T,v)\).  \emph{Hints:} First differentiate to find the
relationship between \(u\) and \(T\).

\item Recall the example done in class of a piston separating two chambers of volumes 1 and
\SI{10}{L}, respectively, each containing 1 mole of ideal gas, the whole system held isothermal at
\SI{273}{K}.  We showed in class that \SI{-2.5}{kJ} of work could be extracted by allowing
the piston to adjust the volumes to \SI{5}{L} and \SI{6}{L}.  Compare the amount of work available
from the same process and conditions if the fluid was van der Waals \ce{CO2}.

\item Compare the ideal and van der Waals \ce{CO2} cases if the volumes started at 0.1 and
\SI{1.0}{L} and ended at 0.5 and \SI{0.6}{L}.

\item Take the same case of a system of total volume of \SI{1.1}{L}, but now put 1 mole of van
der Waals \ce{CO2} fluid on one side of the piston and 1 mole of ideal gas on the other.  If
the piston was allowed to freely move, where would it end up?
\end{enumerate}

\section{Manipulating thermodynamic derivatives}
\label{sec:org429086d}
As Prateek might say, the susceptibilities, \(c_P\), \(\alpha\), and \(\kappa_T\), are
  ``awesome.'' Everything about a one component fluid can be expressed in
  terms of them, and for good reason! Here's some practice with them.
\begin{enumerate}
\item Express the derivative \(\left ( \frac{\partial T}{\partial v} \right )_h\) in terms
of the three susceptibilities.

\item Determine the susceptibilities of a van der Waals fluid.

\item Plot \(\left ( \frac{\partial T}{\partial v} \right )_h\)  vs.~\(v\) for \(T\) from \(-50\) to
\SI{50}{\celsius} in \SI{20}{\celsius} increments for van der Waals \ce{CO2}. Under what
conditions of temperature and volume is the derivative negative?
\end{enumerate}

\section{Practice with potentials}
\label{sec:orgf7bc301}
While a Carnot refrigerator sure would be efficient, in practice we don't
  like to wait an eternity for our 'fridges to get cold.  Practical refrigerators use the
  Joule-Thomson effect and irreversible, isenthalpic expansion to create the cold we
  love.  Let's see how cold we can make things with van der Waals \ce{CO2}.

\begin{enumerate}
\item Construct the enthalpy \(h_\text{vdW}(T,v)\) of a van der Waals fluid.

\item Determine the change in temperature when van der Waals \ce{CO2} is isenthalpically
expanded from \(10^{-4}\) to \(10^{-3}~\text{m}^3~\text{mol}^{-1}\), starting from \SI{10}{\celsius}.

\item A more reliable estimate of the change in temperature can be gotten from thermodynamic properties charts.  Thermodynamic charts for \ce{CO2} refrigerant are available at \href{https://i0.wp.com/emersonclimateconversations.com/wp-content/uploads/2015/04/co2-post-3-figure-2.jpg?ssl=1}{link}.  Use the pressure-enthalpy chart at the end of this document to determine the change in temperature for the same isenthalpic expansion as in the previous question.
\end{enumerate}

\section{A departure from ideality}
\label{sec:org6d06299}
The examples above assume the heat capacity of \ce{CO2} to be a constant. More
generally heat capacities are functions of temperature and pressure. (We'll learn why when
we study statistical mechanics.)  It's common to report heat capacities in a hypothetical
ideal gas, pressure-independent limit, \(c_{p}^\text{ig}(T)\).  For instance, the molar heat
capacity of ideal \ce{CO2} is reported to be:
\begin{equation*}
  c_p^\text{ig}(t) = -11.401074 - 55.231532t+5.149108t^2-0.29158t^3+0.110128t^{-2}+115.93493t^{1/2}
\end{equation*}
where \(t=T(K)/1000\).

\begin{enumerate}
\item Find the change in molar entropy of ``ideal gas'' \ce{CO2} when heated from 290 to
\SI{350}{K}.

\item To find the real change in entropy, we would need to know the difference in entropy
between the ideal gas and real states of \ce{CO2} at identical \(T\) and \(v\) conditions.  This
difference is called a ``departure function.'' To create this, we construct a hypothetical
path between the real and ideal gas states, first expanding the van der Waals gas
from \(v\) to \(v\rightarrow \infty\), where it behaves ideally, and then compressing from \(\infty\)
to \(v\) as an ideal gas.  Express \(\left ( \frac{\partial S}{\partial v}\right )_{T}\) in terms
\end{enumerate}
of susceptibilities.

\begin{enumerate}
\item Construct \(\left ( \frac{\partial S}{\partial v}\right)_{T}\) for ideal \ce{CO2}.

\item Construct \(\left ( \frac{\partial S}{\partial v}\right)_{T}\) for van der Waals
\ce{CO2}.

\item Construct the departure function \(S^{vdW}(T,v) - S^{ig}(T,v)\) by appropriate integration.  \emph{Hint}: This difference is equivalent to \emph{adding} \(S^{vdW}\) integrated along the path \(v \rightarrow\infty\) and \(S^{ig}\) integrated along the path \(\infty \rightarrow v\).

\item Use the departure function (twice) to compute the molar entropy change when van der Waals
\ce{CO2} is heated from \SI{290}{K} and \SI{1.0}{L/mol} to \SI{350}{K} and \SI{0.15}{L/mol}.
\end{enumerate}
\end{document}