% Created 2017-11-12 Sun 16:06
% Intended LaTeX compiler: pdflatex
\documentclass[11pt]{article}
\usepackage[utf8]{inputenc}
\usepackage{lmodern}
\usepackage[T1]{fontenc}
\usepackage{fixltx2e}
\usepackage{graphicx}
\usepackage{longtable}
\usepackage{float}
\usepackage{wrapfig}
\usepackage{rotating}
\usepackage[normalem]{ulem}
\usepackage{amsmath}
\usepackage{textcomp}
\usepackage{marvosym}
\usepackage{wasysym}
\usepackage{amssymb}
\usepackage{amsmath}
\usepackage[version=3]{mhchem}
\usepackage[numbers,super,sort&compress]{natbib}
\usepackage{natmove}
\usepackage{url}
\usepackage{minted}
\usepackage{underscore}
\usepackage[linktocpage,pdfstartview=FitH,colorlinks,
linkcolor=blue,anchorcolor=blue,
citecolor=blue,filecolor=blue,menucolor=blue,urlcolor=blue]{hyperref}
\usepackage{attachfile}
\usepackage[left=1in, right=1in, top=1in, bottom=1in, nohead]{geometry}
\geometry{margin=1.0in}
\usepackage{hyperref}
\usepackage{amsmath}
\usepackage{graphicx}
\usepackage{epstopdf}
\usepackage{fancyhdr}
\pagestyle{fancy}
\fancyhf{}
\usepackage[labelfont=bf]{caption}
\usepackage{setspace}
\setlength{\headheight}{10.2pt}
\setlength{\headsep}{20pt}
\renewcommand{\headrulewidth}{0.5pt}
\renewcommand{\footrulewidth}{0.5pt}
\lfoot{\today}
\cfoot{\copyright\ 2017 W.\ F.\ Schneider}
\rfoot{\thepage}
\chead{\bf{Advanced Chemical Engineering Thermodynamics (CBE 60553)\vspace{12pt}}}
\lhead{\bf{Homework 6}}
\rhead{\bf{Due November 22, 2017}}
\usepackage{titlesec}
\titlespacing*{\section}
{0pt}{0.6\baselineskip}{0.2\baselineskip}
\title{University of Notre Dame\\Advanced Chemical Engineering Thermodynamics\\(CBE 60553)}
\author{Prof. William F.\ Schneider}
\usepackage{siunitx}
\usepackage[version=3]{mhchem}
\def\dbar{{\mathchar'26\mkern-12mu d}}
\setcounter{secnumdepth}{3}
\author{William F. Schneider}
\date{\today}
\title{CBE 60553 Homework}
\begin{document}

\begin{OPTIONS}
\end{OPTIONS}

\noindent \textbf{Solve each problem on separate sheets of paper, and clearly indicate the problem number and your name on each.  Carefully and neatly document your answers.  You may use a mathematical solver like Matlab or Mathematica. Use plotting software for all plots.}

\section{It's all a balance}
\label{sec:org7078f63}
The van der Waals equation of state captures the balance between molecular attractions and repulsions that characterize a real fluid. The Helmholtz free energy of a monatomic van der Waals fluid can be written
\begin{equation*}
  a_\text{vdW} = \left \{ - RT \ln (v-b) -1.5 R T \ln (R T) \right\} +\left \{ RT -a/v
  \right \}
\end{equation*}
\noindent where the terms in the first bracket correspond to the entropic repulsive forces and the
terms in the second bracket the energetic attractions.

\begin{enumerate}
\item Plot the repulsive, attractive, and total Helmholtz energies of \ce{CO2} at \SI{280}{K}
vs.~log molar volume from 0.04 to \SI{1}{\liter\per\mole}.  The \ce{CO2} van der Waals constants
are \(a =\SI{3.6551}{\liter\squared\bar\per\mole\squared}\) and \(b = \SI{0.042816}{\liter\per\mole}\).

\item \label{q:Z} Plot the compressibility of van der Waals \ce{CO2} vs.~reduced pressure  \(P_r\) of \(0.1
   < P_r < 10\) at \(T_r =1.05\). The critical temperature of \ce{CO2} is \(T_c =
   \SI{304.2}{K}\) and critical pressure \(P_c = \SI{7.376e6}{\pascal}\).  \emph{Hint}: You will
have to solve a cubic numerically.  Consider which of the three roots is the relevant
one.

\item The compressibility of an ideal gas is \(Z_\text{ig}=1\).  Explain in terms of
microscopic interactions why \(Z\) is greater or less than 1 for the various values of
\(P_r\) in your Question \ref{q:Z} plot.
\end{enumerate}

\section{We had to talk about it at some point}
\label{sec:org403ab60}
The fugacity \(f(T,P)\) of a gas is defined as the function that satisfies

\[
\mu(T,P) =
\mu^\circ(T) + RT \ln \left ( f(T,P)/ P^\circ) \right )
\]
where \(\mu^{\circ}(T)\) is the chemical potential of the fluid in an ideal gas reference
state at reference pressure \(P^{\circ}\). The fugacity has units of pressure.

\begin{enumerate}
\item What is the \(\lim_{P\rightarrow 0} \mu(T,P)\)?  \emph{Hint}: Remember that all gas are ideal in
the limit of zero pressure or infinite volume.

\item What is the \(\lim_{P\rightarrow 0} f(T,P)\)?  Why is fugacity a useful concept?

\item Derive a relationship between the residual volume, \(v_{res} = v - v_{ig}\), and
compressibility \(Z\).

\item Derive the following relationship between fugacity and compressibility.  Use the Gibbs-Duhem relationship between \(d\mu\) and the other intensive variables.
\begin{equation}
  \left ( \frac{\partial\ln\left(f/P\right)}{\partial P} \right )_{T} = \frac{v}{RT} \left ( \frac{Z-1}{Z} \right )
\end{equation}

\item Because equations of state are usually rational functions in \(v\), it is generally
easier to compute \(f(T,v)\) than \(f(T,p)\) Use the chain rule to relate the derivative
above to a derivative in \(v\). Evaluate the expression you get assuming the fluid
follows a one-parameter virial equation of state.

\item Now compute \(\ln (f/P)\) for the one-parameter virial equation of state by integrating
from the ideal gas limit (\(v \rightarrow \infty\) to \(v\)).  Recall that \(f\rightarrow P\)
in the ideal gas limit.

\item This ratio \(f/P\) is called the fugacity coefficient, \(\phi\). When \(\phi > 1\), the
chemical potential is greater than that of an ideal gas at the same density and
temperature, and vice versa when \(\phi < 1\).  Recall we learned that the second virial
coefficient \(B > 0\) at high temperatures, where entropy effects dominate, and \(B < 0\)
at low temperatures, where the energetic interactions between molecules dominate.  How
does the virial EOS chemical potential compare to the ideal gas chemical potential at high
temperature? At low temperature?  Why?

\item One can do similar calculations for more complex equations of state. Be thankful I
didn't ask you to. They are a mess. If you need the results, look them up.
\end{enumerate}

\section{Separating an ideal mixture}
\label{sec:org4f01927}
The exhaust from a coal-fired power plant contains approximately 12$\backslash$% \ce{CO2} in a
  mixture of other gases, all at \SI{40}{\celsius} and \SI{1}{atm}.

\begin{enumerate}
\item What is the minimum work, in kJ/kg \ce{CO2}, to separate the \ce{CO2} from the
remaining gases at constant \(T\) and \(P\), assuming the gas mixture is ideal?

\item What is the minimum work of separation if the mixture is non-ideal and obeys the
Lewis-Randall mixing rules, \(\hat{f}_i^{\text{mix}} = y_i f_i\)?
\end{enumerate}

\[
W_{min} = -\Delta G_{mix} = (N_A \mu_A(y_A,T) + N_B \mu_B(y_B,T)) - N_A \mu_A(T) - N_B \mu_B(T)
\]

\section{And fugacity makes a come-back}
\label{sec:org3825df8}
The virial equation of state can be generalize for a mixture to

\begin{equation*}
  \frac{P}{RT} = \frac{N}{V} + \frac{N^2 B_\text{mix}}{V^2} \qquad B_\text{mix} =
  \sum_i\sum_j y_iy_j B_{ij} \qquad B_{ij} = \sqrt{B_iB_j}
\end{equation*}

\begin{enumerate}
\item Derive an expression for the partial molar volume of component \(\bar v_{i}\) in the virial
mixture.  \textit{Hint:} Take advantage of the triple product rule of partial derivatives.

\item The component fugacity \(\hat{f}_{i}(T,P)\) and the corresponding fugacity coefficient \(\hat{f}_i/y_i
   P = \phi_i(T,P)\) can be defined in analogy to the expression for the pure component system.
Use this definition to derive an expression for \(\ln \phi_i\) for the virial
mixture, integrating as appropriate from the ideal gas limit to \(P\).
\begin{equation}
  \left ( \frac{\partial\ln\left(\hat{f}_{i}/y_{i}P\right)}{\partial P} \right )_{T} = \frac{\bar{v}_{i} - \bar{v}_{ig}}{RT}
\end{equation}

\item (4 pts) The second virial coefficients of \ce{CO2} and air are \(-110.7\) and
\(-3 \text{ cm}^3 \text{ mol}^{-1}\) at \(40^\circ\)C. What is the minimum work of
separation of 12$\backslash$% \ce{CO2} from \ce{N2} approximating the mixture with the
virial equation of state?
\end{enumerate}
\end{document}